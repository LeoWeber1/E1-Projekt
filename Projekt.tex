\documentclass[a4paper,12pt]{article}
\usepackage[utf8]{inputenc}
\usepackage[T1]{fontenc}
\usepackage[english]{babel} % English language settings and hyphenation
\usepackage{enumitem}

% Modern font
\usepackage[sfdefault,light]{roboto} % Clean and modern sans-serif font
%\usepackage[sfdefault]{noto} % Alternative: Noto Sans, also modern

\usepackage{geometry}
\geometry{margin=2.5cm} % Page margins

\usepackage{graphicx}
\usepackage{caption}
\usepackage{tocloft} % Table of contents customization
\usepackage{parskip}    % Optional, improves readability
\setlength{\parindent}{0pt}  % Disable paragraph indentation
\graphicspath{{./assets/}} % Path relative to the main .tex file

\usepackage[hyphens]{url} % Clickable URLs with hyphen line breaks
\usepackage{amsmath} % For \text and math environments
\usepackage{amsfonts} % For math symbols
\usepackage{amssymb} % For \checkmark

\usepackage{listings} % For formatted code blocks
\usepackage{xcolor} % For colors (e.g. gray border)
\usepackage{float} % For precise figure placement with [H]

% Configuration for listings (terminal commands)
\lstset{
	basicstyle=\ttfamily\small,
	breaklines=true,
	breakatwhitespace=true,
	frame=single,
	framerule=0.5pt,
	rulecolor=\color{gray},
	xleftmargin=0pt,
	xrightmargin=0pt,
	aboveskip=7pt,
	belowskip=5pt,
	columns=fullflexible,
	keepspaces=true,
	showspaces=false,
	showstringspaces=false
}

% Optional: More elegant section titles
\usepackage{titlesec}
\titleformat{\section}{\normalfont\Large\bfseries\sffamily}{\thesection}{1em}{}
\titleformat{\subsection}{\normalfont\large\bfseries\sffamily}{\thesubsection}{1em}{}


% Configuring the title page
\title{\sffamily\bfseries Project Documentation: Virtual Safety and A4P Safety Education Kit}
\author{\sffamily Project Team}
\date{\sffamily \today}


\begin{document}

\maketitle
\clearpage

\tableofcontents
\clearpage

\section{Introduction}
The project "Virtual Safety and A4P Safety Education Kit" aims to integrate the "Virtual Safety" functionality of CODESYS, evaluate its application within the "Allgäu 4 Production (A4P)" platform, and develop a Safety Education Kit.

\section{Hardware Requirements}
The following hardware components are required for the project:
\begin{itemize}
\item 1 router
\item 2 computers with Debian Linux installed  
Preferably with 2 LAN ports, otherwise an adapter can be used.
\item 1 Windows PC
\end{itemize}

\section{Debian Installation}
The following steps describe the installation of Debian:

\subsection{Preparing the Installation Medium}
\begin{enumerate}
\item Download the Netinst CD image for amd64 from \url{https://www.debian.org/CD/netinst/}.
\item Create a bootable USB stick using Rufus (\url{https://rufus.ie/en/}). Download the portable .exe file and use the default settings to flash.
\end{enumerate}

\subsection{BIOS Configuration and Starting the Installer}
\begin{enumerate}
\item Connect a network, keyboard, and monitor to the IPC/PLC.
\item Boot into BIOS and set the USB stick as the primary boot device.
\item Press F10 and confirm settings.
\item The IPC should boot into the graphical installer.
\end{enumerate}

\subsection{Performing the Installation}
\begin{enumerate}
\item The "Debian Net Installer" starts.
\item Select "Graphical Installer" (requires internet connection).
\item Follow the installation (language, keyboard layout, country, etc.).
\item Set up passwords and user accounts.
\end{enumerate}

\subsection{Partitioning the Hard Drive}
\begin{enumerate}
\item Select "Guided - use entire disk".
\item Select "No separate partition".
\item Choose the Ext4 file system.
\item Click "Continue", "Continue", "Continue" $\rightarrow$ "Finish".
\end{enumerate}

\subsection{Mirror Selection}
\begin{enumerate}
\item Select all suggested mirror servers.
\item Do not use a proxy.
\end{enumerate}

\subsection{Software Selection}
\begin{enumerate}
\item Select "SSH Server".
\item Select "Standard System Utilities".
\item Deselect "Desktop Environment" as it is not required.
\item Click "Finish".
\end{enumerate}

\subsection{Reboot and First Steps}
\begin{enumerate}
\item Reboot the system (without the USB stick!).
\item Log in as ROOT user (username: root, password: the one set during installation).
\end{enumerate}

\subsection{Creating a SUDO User}
\begin{enumerate}
\item Install sudo and add the "codesys" user to the sudo group:
\begin{lstlisting}
apt install sudo && adduser codesys sudo
\end{lstlisting}
\item Save and close Nano with Ctrl+X.
\end{enumerate}

\subsection{SSH Configuration}
\begin{enumerate}
\item Open the SSH configuration file:
\begin{lstlisting}
nano /etc/ssh/sshd_config
\end{lstlisting}
\item Uncomment the line by removing the \texttt{\#}:
\begin{lstlisting}
PermitRootLogin yes
\end{lstlisting}
\end{enumerate}

\subsection{Logging Out the ROOT User}
\begin{enumerate}
\item Log out the ROOT user:
\begin{lstlisting}
exit
\end{lstlisting}
\item Log in with the regular user.
\end{enumerate}

\subsection{Installing Python 3}
\begin{enumerate}
\item Install Python 3:
\begin{lstlisting}
sudo apt install python3
\end{lstlisting}
\item Log out:
\begin{lstlisting}
exit
\end{lstlisting}
\end{enumerate}

\subsection{SSH Connection and Docker Installation}
\begin{enumerate}
\item Log in via SSH (open PowerShell in Windows and use the following command) or use Putty and follow the Docker install guide:
\begin{lstlisting}
ssh codesys@hostname
\end{lstlisting}
\item Follow the link to install Docker: \url{https://docs.docker.com/engine/install/debian/}
\end{enumerate}

\newpage
\section{Making Debian Real-Time Capable}
\begin{enumerate}
\item Install the real-time kernel and test tools:
\begin{lstlisting}
sudo apt install linux-image-rt-amd64 rt-tests
\end{lstlisting}

\item Open the GRUB configuration:
\begin{lstlisting}
sudo nano /etc/default/grub
\end{lstlisting}

\item Edit the line \texttt{GRUB\_CMDLINE\_LINUX\_DEFAULT} depending on CPU type:

\subsubsection*{For Intel Systems:}
\begin{lstlisting}
GRUB_CMDLINE_LINUX_DEFAULT="quiet igb.EEE=0 processor.max_cstate=0 \
processor_idle.max_cstate=0 intel_idle.max_cstate=0 clocksource=tsc tsc=reliable \
nmi_watchdog=0 nosoftlockup intel_pstate=disable idle=poll noht rcu_nocb_poll \
hugepages=1024 i915.enable_dc=0 i915.disable_power_well=0 mce=off hpet=disable \
numa_balancing=disable efi=runtime"
\end{lstlisting}

\subsubsection*{For AMD Systems:}
\begin{lstlisting}
GRUB_CMDLINE_LINUX_DEFAULT="quiet idle=poll clocksource=tsc tsc=reliable \
nmi_watchdog=0 nosoftlockup hugepages=1024 rcu_nocb_poll \
numa_balancing=disable efi=runtime"
\end{lstlisting}

\item Optional: Set boot delay to 0:
\begin{lstlisting}
GRUB_TIMEOUT=0
\end{lstlisting}

\item Save the file and regenerate GRUB:
\begin{lstlisting}
sudo update-grub
\end{lstlisting}

\item Reboot the system:
\begin{lstlisting}
sudo reboot
\end{lstlisting}

\item Check if the real-time kernel is active:
\begin{lstlisting}
uname -a
\end{lstlisting}
\textit{The output should contain something like \texttt{PREEMPT\_RT}.}

\item Test the real-time performance:
\begin{lstlisting}
sudo cyclictest -p 99 -t -m
\end{lstlisting}
\textit{Note: A latency under 100 µs is a good value.}

\item Further optimization tips for real-time capabilities can be found at: \\ 
\url{https://confluence.codesys.com/x/AoNZEQ}
\end{enumerate}

\subsubsection*{Explanation of Key GRUB Parameters:}
\begin{itemize}
\item \texttt{idle=poll} – Prevents CPU sleep states (for lower latency).
\item \texttt{clocksource=tsc tsc=reliable} – Uses a stable time source.
\item \texttt{rcu\_nocb\_poll} – Decouples RCU interrupts from certain CPUs.
\item \texttt{hugepages=1024} – Reserves large memory pages.
\item \texttt{nosoftlockup, nmi\_watchdog=0} – Prevents interrupt issues.
\item \texttt{intel\_pstate=disable, noht, processor.max\_cstate=0} – Relevant for Intel CPUs only.
\item \texttt{i915.enable\_dc=0, i915.disable\_power\_well=0} – Only relevant for Intel graphics.
\end{itemize}
\newpage


\section{For Native Linux SL Installation (a SoftSPS)}

The following packages are required to run CODESYS Control directly on a Linux host:

\begin{figure}[H]
	\centering
	\includegraphics[width=1\textwidth]{1.jpg}
	\caption{Overview of CODESYS package Manager. Search for the following Packages in the Search Bar}
	\label{fig:packages-overview}
\end{figure}

\begin{itemize}
	\item \textbf{CODESYS Control for Linux SL} \\
	\textit{Note: ARM devices are currently not supported.}
	\item \textbf{CODESYS Control SL Deploy Tool}
	\item \textbf{CODESYS Edge Gateway for Linux SL}
\end{itemize}



\subsection{For Running Multiple Controllers with a Container Engine (e.g. Docker)}
When running multiple SoftSPS instances on the same host:
\begin{itemize}
	\item \textbf{CODESYS Virtual Control for Linux SL}
	\item \textbf{CODESYS Virtual Safe Control SL}
\end{itemize}

\subsection{For Safety Features with CODESYS}
If safety-relevant controllers are to be used, you additionally need:
\begin{itemize}
	\item \textbf{CODESYS Safety Extension}
	\item \textbf{CODESYS Safe Control Service} \\
	\textit{Only required for systems with Safe Control Core.}
	\item{To integrate PROFINET-based fieldbus systems: CODESYS PROFINET}
\end{itemize}

\newpage

\section{Software Requirements of the Host Device}

The host device must currently meet the following software requirements to ensure proper installation and operation of the necessary CODESYS packages.

\subsection*{Architecture and Compatibility}

\begin{itemize}
	\item The host must use a \textbf{64-bit CPU (AMD64 architecture)}.
	\item The operating system must support the \textbf{i386 (32-bit)} architecture. This is required specifically for using \texttt{CODESYS Control for Linux SL - AMD64} in combination with \texttt{Safe Control}, which requires 32-bit support.
	\item To enable 32-bit support on Debian-based systems, execute the following commands:
\end{itemize}

\begin{lstlisting}
sudo dpkg --add-architecture i386
sudo apt update
sudo apt install libc6-i386
\end{lstlisting}

\subsection*{Operating System Requirements}

\begin{itemize}
	\item The host device should run a \textbf{Debian-based distribution} that supports the \texttt{dpkg} package manager.
	\item RPM-based systems (e.g., Red Hat) can be used as well. As of version \textbf{4.15.0.0}, the Linux Deploy Tool supports manual installation of RPM packages.
\end{itemize}

\subsection*{Real-Time Capability}

\begin{itemize}
	\item The Linux system must provide \textbf{real-time capability}.
	\item For optimal configuration and tuning of the real-time environment, please refer to the official CODESYS online documentation:
	
	\url{https://content.helpmecodesys.com/de/CODESYS%20Control/_rtsl_performance_optimization_linux.html}
\end{itemize}

\newpage
\section{Network Configuration}

For stable operation and correct communication with CODESYS and the Timeprovider, the following network configuration is recommended:

\begin{figure}[H]
	\centering
	\includegraphics[width=1\textwidth]{2.jpg}
	\caption{Recommended network configuration with WLAN and LAN devices}
\end{figure}

\subsection*{Network Structure}

The network consists of a central router that provides both WLAN and wired LAN connections. The individual devices in the network are connected as follows:

\begin{itemize}
	\item \textbf{WLAN:}
	\begin{itemize}
		\item A Windows PC with CODESYS is connected via WLAN.
		\item This device has an assigned IP address and establishes the connection to the CODESYS controller.
	\end{itemize}
	
	\item \textbf{LAN:}
	\begin{itemize}
		\item All LAN devices also have static IP addresses.
		\item These devices are connected via LAN to a \textbf{Timeprovider} to provide a precise time source within the network.
	\end{itemize}
\end{itemize}

Two dedicated systems are required to ensure certification and thus security.

\newpage
\section{Installing Runtime Systems}


With the CODESYS Deploy Tool, a connection can be established to the Linux host where the necessary package-based or container-based runtime systems are to be installed.  
Navigate to:  
\texttt{Tools → Deploy Control SL → Connect to the Linux target system via SSH connection.}


In the \textbf{Deployment} tab, all available package- and container-based runtime system versions and components that can be installed on the device are listed.  
Here, the appropriate package (or container) and its version can be selected and installed on the Linux device.

\begin{figure}[H]
	\centering
	\includegraphics[width=1\textwidth]{3.jpg}
	\caption{Open Deploy Control: shows target device's IP address (WLAN), username ("with root access"), and Linux user password input}
\end{figure}


\section{Deployment to use Control Linux SL (without Docker)}

For the usage of CODESYS Control for Linux SL you have to install the following Packages:

\begin{figure}[H]
	\centering
	\includegraphics[width=1\textwidth]{4.jpg}
	\caption{What you see under "Deployment"}
\end{figure}
\newpage


\section{Installing CODESYS Virtual Control for Linux SL}

This section provides a comprehensive guide for deploying the CODESYS Virtual Control for Linux SL runtime environment, including both a virtual standard PLC (vPLC) and a virtual safety PLC (vSafePLC) with a dedicated Timeprovider. The deployment is split across two separate Linux systems for real-time safety certification and architectural integrity.

\subsection{Deployment Overview (Using Docker Containers)}

To implement this scenario, the following setup is required:

\begin{itemize}
	\item \textbf{Linux Host PC 1} – Hosts the vPLC and vSafePLC runtime systems.
	\item \textbf{Linux Host PC 2} – Hosts the Safety Timeprovider service.
\end{itemize}

An internet connection and Docker must be available on both systems. SSH should be configured to allow remote connection via the CODESYS Deploy Tool.

\vspace{0.5em}
Refer to the official CODESYS documentation for background:
\url{https://content.helpme-codesys.com/en/CODESYS%20Control/_rtsl_scenario_safe_house.html}

\subsection*{Step 1: Install the vSafe Timeprovider on Linux Host PC 2}

\begin{enumerate}
	\item Launch the CODESYS Deploy Tool.
	\item Connect to \textbf{Linux Host PC 2} via \texttt{Tools → Deploy Control SL → Communication tab}.
	\item Switch to the \textbf{Deployment} tab.
	\item Select \texttt{CODESYS Virtual Safe Time Provider SL} from the product list.
	\item Choose the latest available version and click \texttt{Install}.
	\item Switch to the \textbf{Operation} tab and click the \texttt{+} button to add a new instance.
	\item Name the instance \texttt{timeprovider}, select \textbf{Safety Timeprovider} from the filter list, and use the latest version.
	\item In the instance settings:
	\begin{itemize}
		\item Set \texttt{TARGET\_IP} to the IP address of \textbf{Linux Host PC 1}.
		\item Ensure \texttt{TARGET\_PORT} is \texttt{60000}.
		\item Set \texttt{Autostart} to \texttt{Yes}.
	\end{itemize}
	\item Click \texttt{Save}, then \texttt{Start} the instance.
\end{enumerate}

\begin{figure}[H]
	\centering
	\includegraphics[width=1\textwidth]{5.PNG}
	\caption{Deploy Tool – Configuration of Timeprovider instance on Linux Host PC 2.}
\end{figure}

\subsection*{Step 2: Install vPLC and vSafePLC on Linux Host PC 1}

\begin{enumerate}
	\item Connect to \textbf{Linux Host PC 1} via the CODESYS Deploy Tool.
	\item In the \textbf{Deployment} tab, install:
	\begin{itemize}
		\item \texttt{CODESYS Virtual Control SL} (vPLC)
		\item \texttt{CODESYS Virtual Safe Control SL} (vSafePLC)
	\end{itemize}
	\item Go to the \textbf{Operation} tab and click \texttt{+} to add new instances:
	\begin{itemize}
		\item Create \texttt{vPLC} – Select \textbf{Runtime System}.
		\item Create \texttt{vSafePLC} – Select \textbf{Safety Runtime System}.
	\end{itemize}
	\item Configure the \texttt{vSafePLC} instance:
	\begin{itemize}
		\item \texttt{Ports}: \texttt{60000:60000/udp}
		\item \texttt{IPC}: \texttt{container:vPLC}
		\item \texttt{Dependencies}: \texttt{vPLC} must start first
	\end{itemize}
	\item Configure the \texttt{vPLC} instance:
	\begin{itemize}
		\item Enable \texttt{shareable} IPC namespace
	\end{itemize}
	\item Click \texttt{Start All} to run both instances.
\end{enumerate}

\begin{figure}[H]
	\centering
	\includegraphics[width=1\textwidth]{6.PNG}
	\caption{VGateway Configurations. Feel free to copy the Mounts}
\end{figure}

\begin{figure}[H]
	\centering
	\includegraphics[width=1\textwidth]{7.PNG}
	\caption{VPLC Configurations}
\end{figure}

\begin{figure}[H]
	\centering
	\includegraphics[width=1\textwidth]{8.PNG}
	\caption{Use the same Mounts}
\end{figure}

\subsection*{Step 3: Verify the Time Synchronization}

To confirm proper communication between the Timeprovider and vSafePLC:

\begin{itemize}
	\item In the Deploy Tool, select the \texttt{vSafePLC} instance.
	\item Open the log via the \texttt{Show Log} action in the top-right corner.
	\item Check for the message: \texttt{External Time Provider found}.
\end{itemize}

If this message is present, the safety time synchronization is functioning correctly.

\vspace{0.5em}
All runtime systems are now installed and running. You can proceed to create your project and integrate PROFIsafe.

\newpage


\section{CODESYS Project for Using CODESYS Safe Control}
\label{sec:safe_control_project}

\subsection{Importing Devices}
To automatically generate the logical safety devices in the device tree, a specific import option must be enabled in the PROFINET plugin.

After enabling this setting, re-import the GSDML file for the corresponding PROFINET fieldbus devices.

\begin{figure}[H]
	\centering
	\includegraphics[width=0.8\textwidth]{9.JPG}
	\caption{VPLC Configurations}
\end{figure}

\textbf{Note:} Ensure the following option is enabled: \\
\texttt{Options → PROFINET → Create logical devices for safety addon}

\subsection{Creating a New Project}
\textbf{Note:} Using the \textit{Empty Safety Project} template will automatically enable user management for the safety project.

Alternatively, you can use the \textit{Empty Project} or \textit{Standard Project} templates.  
If using the \textit{Standard Project}, select the following device:  
\texttt{CODESYS Control for Linux SL (CODESYS)}

\begin{figure}[H]
	\centering
	\includegraphics[width=0.8\textwidth]{10.JPG}
\end{figure}

In the project tree, right-click on the device and select:  
\texttt{Add Device → PLCs},  
then choose the appropriate \texttt{CODESYS Safe Control} device to add it to your project.

\begin{figure}[H]
	\centering
	\includegraphics[width=0.6\textwidth]{11.JPG}
\end{figure}

\subsection{Project Tree Overview}
After adding the \texttt{CODESYS Safe Control for Linux SL} controller, the project tree now includes:

\begin{itemize}
	\item A dedicated \textbf{Safety Logic}
	\item A \textbf{Safety Application}
	\item A separate \textbf{Library Manager}
	\item Logical \textbf{I/Os}
	\item A defined \textbf{Safety Task}
\end{itemize}

\begin{figure}[H]
	\centering
	\includegraphics[width=0.5\textwidth]{12.JPG}
\end{figure}

These components provide the foundation for developing and managing the safety-related aspects of your application in CODESYS.

\subsection{Communication with Linux Runtime Systems}

To establish communication between the CODESYS development environment and the Linux runtime system, a gateway is required. This can either be a local gateway or a \textbf{CODESYS Edge Gateway for Linux} installed directly on the target device.

\begin{figure}[H]
	\centering
	\includegraphics[width=1\textwidth]{gateway.png}
\end{figure}

Initial setup instructions for installing the gateway and establishing communication with a Linux runtime can be found here:

\begin{quote}
	\url{https://content.helpme-codesys.com/en/CODESYS%20Control/_rtsl_load_and_start_application.html}
\end{quote}

\subsection{Installing a Safe Timeprovider}

A \textbf{Safe Timeprovider} is always required for the operation of hardware-independent Safe Control runtime systems. This component is available as a separate software package and provides essential timing information for safety-related applications.

For improved fault detection, the timeprovider should run on a \textbf{second device} with a different CPU. It periodically sends a timestamp to the device running the safety controller, allowing it to compare and detect discrepancies in CPU clock timing—an important aspect of safety certification.

\vspace{0.5em}
\textbf{Development and Testing:}  
For initial testing, offline programming, or virtual commissioning, the timeprovider may also be installed on the \textbf{same device} as the safety controller.

\vspace{0.5em}
\textbf{Production Systems:}  
For deployment in a certified application, the timeprovider \textbf{must be installed on separate hardware} to ensure safety integrity.

\vspace{0.5em}
If you are using a package-based controller, the \texttt{Safe Timeprovider SL} can be deployed via the \textbf{CODESYS Deploy Tool}.

\paragraph{Same-Device Configuration:}  
If the Safe Timeprovider runs on the same machine as the \texttt{CODESYS Safe Control}, no additional environment variable is required. The time signal is sent by default to:

\begin{itemize}
	\item \texttt{localhost} on port \texttt{9000}
\end{itemize}


\begin{figure}[H]
\centering
\includegraphics[width=1\textwidth]{14.JPG}
\end{figure}

\paragraph{Two-Device Configuration:}  
If the Safe Timeprovider runs on a separate Linux device, you must set the target IP address as an environment variable on that system:

\begin{lstlisting}[language=bash]
export TARGET_IP=XXXX.XXXX.XXX.XXXX
\end{lstlisting}

In the \texttt{Safe Control} log, a successful connection to the timeprovider should be visible before performing a download. This indicates that synchronization is working correctly.

\begin{figure}[H]
\centering
\includegraphics[width=1\textwidth]{15.JPG}
\end{figure}

\paragraph{Changing the Time Cycle:}  
The default sending cycle is \texttt{10 ms}. You can override this setting using the \texttt{CYCLETIME} environment variable. For example, to change the cycle time to \texttt{3 ms}:

\begin{lstlisting}[language=bash]
export CYCLETIME=3
\end{lstlisting}


\subsection{Example: Adding a Safe Application Program}

% Introducing the example of creating a simple counter in a safe application
As an introductory example, this section illustrates how a simple counter can be created within a safe application.

% Adding a new program using the SafetyApp context menu
By right-clicking on the \texttt{SafetyApp} and selecting \texttt{Add Object} $\rightarrow$ \texttt{Extended POU (Safety)}, a new program can be added.

\begin{figure}[H]
	\centering
	\includegraphics[width=0.5\textwidth]{e1.JPG}
\end{figure}

% Choosing between a program or function block for the POU
For the POU, a selection can be made between a program or a function block.

\begin{figure}[H]
	\centering
	\includegraphics[width=0.4\textwidth]{e2.JPG}
\end{figure}

% Adding a network and an adder block
Within the program, a network is already defined. Using the toolbox, an \texttt{ADD} adder block with two inputs can be added.
\begin{figure}[H]
	\centering
	\includegraphics[width=1\textwidth]{e3.JPG}
\end{figure}
% Assigning variables to the adder block inputs
A variable named \texttt{counter} can be added to one input. The second summand can be added to the second input.
\begin{figure}[H]
	\centering
	\includegraphics[width=0.8\textwidth]{e4.JPG}
\end{figure}
% Highlighting changes in a Safe POU
Changes in a Safe POU are marked in red, and this does not indicate an error.
% Adding and deleting a network to clear the red highlight
Finally, a network can be added and subsequently deleted, ensuring the POU is no longer marked in red.
\begin{figure}[H]
	\centering
	\includegraphics[width=1\textwidth]{e5.JPG}
\end{figure}
% Automatic assignment of the POU to the Safety Task
After adding the POU to the project tree, it is automatically assigned to the Safety Task.
\begin{figure}[H]
	\centering
	\includegraphics[width=1\textwidth]{e6.JPG}
\end{figure}
% Connecting to the Linux runtime system and downloading the project
Subsequently, you can connect to the Linux runtime system and download the project to the target device. 
\begin{figure}[H]
	\centering
	\includegraphics[width=1\textwidth]{e7.JPG}
\end{figure}
As previously described, the Safe Control log can be checked to confirm the connection to the timer.

% Selecting the SafetyApp as the active application
The \texttt{Safe Control SafetyApp} can then be selected as the active application. Via the communication dialog of the device, you can log in to the safety controller.
\begin{figure}[H]
	\centering
	\includegraphics[width=1\textwidth]{e8.JPG}
\end{figure}
% Handling the download type prompt
After logging in, a yellow dialog window will prompt for the type of download.
\begin{figure}[H]
	\centering
	\includegraphics[width=0.7\textwidth]{e9.JPG}
\end{figure}
For a brief test of the counter, a temporary download can be performed. A window will then open, requesting a password.

% Handling the password prompt for the download
Since no password has been defined on the controller yet, the field can be left empty and confirmed by clicking \texttt{OK}.
\begin{figure}[H]
	\centering
	\includegraphics[width=0.7\textwidth]{e10.JPG}
\end{figure}
% Running and verifying the application
The application is then downloaded and in the \texttt{Stop} state. After starting the application, the \texttt{counter} variable in the POU should cyclically increment. 
\begin{figure}[H]
	\centering
	\includegraphics[width=1\textwidth]{e11.JPG}
\end{figure}
During a normal download (DL), the program still runs in the unsafe state (DL).
\begin{figure}[H]
	\centering
	\includegraphics[width=1\textwidth]{e13.JPG}
\end{figure}
% Creating a boot application for safe operation
To run the program in a safe state, a boot application must be created on the safety controller. Creating the boot application requires logging out of the safety controller, which unloads the download.

% Selecting and sealing the boot application
\begin{figure}[H]
	\centering
	\includegraphics[width=0.8\textwidth]{e14.JPG}
\end{figure}
The boot application is then selected. You will be prompted to confirm whether you want to create the boot application, which must be confirmed with \texttt{Yes}. 
\begin{figure}[H]
	\centering
	\includegraphics[width=0.8\textwidth]{e15.JPG}
\end{figure}
Subsequently, you will receive confirmation that the boot application has been created on the device. The application is now in the \texttt{Stop} state again.
\begin{figure}[H]
	\centering
	\includegraphics[width=1\textwidth]{e16.JPG}
\end{figure}
 After starting the application, it remains in the unsafe state but with a boot application
\begin{figure}[H]
	\centering
	\includegraphics[width=1\textwidth]{e17.JPG}
\end{figure}

To enter the safe state during the boot application, a logout is necessary along with the confirmation of the restart of the boot application.
\begin{figure}[H]
	\centering
	\includegraphics[width=0.8\textwidth]{e18.JPG}
\end{figure}
\begin{figure}[H]
	\centering
	\includegraphics[width=0.8\textwidth]{e19.JPG}
\end{figure}
Now the application has been restarted.
\begin{figure}[H]
	\centering
	\includegraphics[width=0.8\textwidth]{e19e20.JPG}
\end{figure}
To view the status of the boot app, the Safety Online Information dialog from the Safe Control can be accessed.

\begin{figure}[H]
	\centering
	\includegraphics[width=0.8\textwidth]{e21.JPG}
\end{figure}
After logging in again to the safety controller, a confirmation to start the boot app is requested in the log messages from the Safety Runtime.
\begin{figure}[H]
	\centering
	\includegraphics[width=0.8\textwidth]{e22.JPG}
\end{figure}
The application is now in the safe state but has not yet started.
\begin{figure}[H]
	\centering
	\includegraphics[width=1\textwidth]{e23.JPG}
\end{figure}
The start of the boot app is performed via the non-safe standard controller. For this purpose, a program was created that implements the \texttt{SafeControl.StartBootApp} FB.

\begin{figure}[H]
	\centering
	\includegraphics[width=1\textwidth]{e24.JPG}
\end{figure}
Code:
\begin{verbatim}
	PROGRAM Starting_SafetyApp_BootApp
	VAR
	xStartBA : BOOL := FALSE;
	xRestart : BOOL := FALSE;
	
	fbSafeApp  : SafeControl.SafeApplication;
	fbSafeDev  : SafeControl.SafeDevice;
	
	fbStartBA  : SafeControl.StartBootApp;
	fbRestart  : SafeControl.RestartBootApp;
	END_VAR
	
	------------------------------------------------------------
	
	VAR CONSTANT
	c_udnClientId : UDINT := 16#ED387206;   // The given client id of the standard PLC
	END_VAR
	
	fbSafeApp(xEnable := TRUE);
	fbSafeDev(xEnable := TRUE);
	
	fbStartBA(xExecute := (fbSafeApp.xBootAppConfirmationRequested AND xStartBA),
	szSafetyDeviceFirmware := fbSafeDev.szSafetyDeviceFirmware,
	udnAppId := fbSafeApp.AppInfo.udnAppId,
	udnClientId := c_udnClientId,
	BootAppConfirmation := fbSafeApp.BootAppConfirmation);
	
	// For automatic test execution FB with the restart of the bootapplication is also implemented
	fbRestart(xExecute := xRestart);
\end{verbatim}

\newpage
With the variable \texttt{xStartBA}, the boot application can be manually activated.
\begin{figure}[H]
	\centering
	\includegraphics[width=1\textwidth]{e25.JPG}
\end{figure}
Subsequently, the Safe Control is started and is in the safe state (BA).
\begin{figure}[H]
	\centering
	\includegraphics[width=1\textwidth]{e26.JPG}
\end{figure}


\subsection{Location of the Configuration File and Client ID for Safety Runtime}

For the boot application confirmation, a \textbf{Client ID} is required. The configuration file of the Safe Control runtime system contains a default Client ID.

It is also possible to define custom Client IDs individually in this configuration file.

For the package-based Safe Control system, the configuration file can be found in the following directory on the Linux host:

\begin{verbatim}
	/etc/codesyssafecontrol
\end{verbatim}

\begin{figure}[H]
	\centering
	\includegraphics[width=1\textwidth]{16.JPG}
\end{figure}

The log files for the Safe Control runtime system are stored in:

\begin{verbatim}
	/var/opt/codesyssafecontrol
\end{verbatim}

\subsection{Renaming the Safe Control Device}

Renaming the safety controller is possible via the communication settings:

\texttt{Device → Active Device → Rename}
\begin{figure}[H]
	\centering
	\includegraphics[width=0.5\textwidth]{17.JPG}
\end{figure}


\section{Logical Devices}

\subsection{Data Exchange Between Safety and Standard Controller}

To set up data exchange between the safety and standard controller, follow these steps:

\begin{enumerate}
	\item In the safety controller:
	\begin{itemize}
		\item Navigate to \texttt{<Safety-Device> → Safety Logic → SafetyApp → Logical I/Os}.
		\item Open the context menu and select the command \textbf{Add Logical Device}.
		\item In the dialog, select an object from the \textbf{Logical Exchange Devices} node.
	\end{itemize}
	\begin{figure}[H]
		\centering
		\includegraphics[width=0.6\textwidth]{18.JPG}
	\end{figure}
	
	\item In the standard controller:
	\begin{itemize}
		\item Navigate to \texttt{<Standard-Device> → PLC Logic → Application}.
		\item Open the context menu and select \textbf{Add Object → Logical Exchange GVL}.
		\item Give the object an appropriate name and add it to the project tree.
	\end{itemize}
		\begin{figure}[H]
		\centering
		\includegraphics[width=0.7\textwidth]{19.JPG}
	\end{figure}
	
	\item In the editor:
	\begin{itemize}
		\item Link the logical exchange device with the GVL (Global Variable List).
	\end{itemize}
\end{enumerate}


\section{Logical Devices}

\subsection{Data Exchange Between Safety and Standard Controller}

In the node \texttt{-> Safety Logic -> SafetyApp -> Logical I/Os}, open the context menu and select the command \textbf{"Add Logical Device"}. In the dialog, select an object from the node \texttt{Logical exchange devices}.

In the node \texttt{-> PLC Logic -> Application}, open the context menu and select \textbf{"Add Object -> Logical Exchange GVL"}, give it an appropriate name, and add it to the project tree.

In the editor, the logical exchange device can be linked with the GVL.

\subsection*{Exchange via Fieldbus with Safety IOs}

The exchange of IO data also occurs via logical devices. If the assignment is clear (e.g., only one safety controller in the project tree), then when inserting the physical device, the logical device is automatically inserted and linked under the Safety Application in the node \texttt{Logical I/Os}.

The linkage is shown in the project tree for both the physical and logical device after the name in \texttt{[->]} or \texttt{[<-]}, and in the mapping editor of the physical device.

In the mapping editor of the physical device, the link can be reset or re-established.

A global FB instance is created in the Safety Application for the inserted logical device with the variable name and type corresponding to the logical device. The FB instance is clearly defined in the Safety Mapping Editor of the logical device.

\textbf{Note:} When inserting, the logical device name may begin with an "\_", whereas the FB instance begins with "x\_" (only from Safety Version 4.2 onwards; older versions will produce a build error).

\textbf{SIL3-supported fieldbuses with safe IOs:}
\begin{itemize}
	\item PROFInet with PROFIsafe V2.4
	\item PROFInet with PROFIsafe V2.6 (only Safe Control Core)
	\item EtherCAT with FSoE (planned for Safe Control Core)
\end{itemize}

\section{PROFIsafe}

Two versions: PROFIsafe V2.4 and V2.6. New F-Device devices may only support V2.6. The physical device defines the version; it cannot be changed.

\section{Safety Application}

Each object corresponds to a "yellow" editor. Each object editor stores the data 1:1 in the "interpreter" format, which is loaded into the safety runtime during download.

The safety runtime also expects information from the safety checker, which is executed during "Build" or before "Login with Download".

\subsection{Global Variable List (GVL)}

\textbf{Note:} No namespace in the GVL. Variables are accessed via \texttt{VAR EXTERNAL}. No problem, as all variable names in a safety application must be unique.

\subsection{IO Mapping}

Definition of variables mapped to the safety application. Variable definition is implicitly global. Use in safety application with \texttt{VAR\_EXTERNAL}.

In the mapping editor, both the link to the physical device and the global FB instance of the corresponding IO stack are shown.

\textbf{Note:} A byte must be defined either as a byte or as individual bits—mixing is not allowed.

\subsection{IO Configuration (F-Parameter)}

Settings according to the assigned physical safe I/O device.

For PROFIsafe or FSoE F-devices: F-Destination Address with DIP switches on the IO or address assignment using vendor-specific tools.

\subsection{POU, FB}

Two different object types:
\begin{itemize}
	\item Basic: Only FBs using boolean logic combinations with AND or OR. No NOT!
	\item Extended: Full set of supported operators (see ToolBox).
\end{itemize}

\textbf{Note:} Safety programming strictly distinguishes between logical and numerical data types.

\begin{itemize}
	\item Logical types: \texttt{BOOL, BYTE, WORD, DWORD} with operations AND, OR, NOT
	\item Numerical types: \texttt{INT, UINT, DINT} with operations ADD, SUB, DIV, MUL, LE, GT, etc.
\end{itemize}

Structures, arrays, enums, and pointers are not supported.

\subsection{Safety Task}

Only one cyclic task with the specified cycle time (default: 10 ms).

Programs listed in the task are executed in the specified order. The execution order can be changed using Up/Down or by selecting (All/None).

\section{Diagnostics}

\subsection{Exchange Between Safety and Standard}

Configuration ID via the module list of exchanged devices (\texttt{IoDrvSafetySP}). Display in the device tree on the safety controller shows whether the ID matches or not.

\textbf{Note:} Changing the PROFIsafe configuration (e.g., F\_WD\_Time) requires a standard download, as the CRC over the F-Parameters is part of the Configuration ID.

\subsection*{IO Stack Instance}

See online help: \url{https://content.helpme-codesys.com/en/CODESYS%20Safety%20Extension/sil3_field_buses.html}

\textbf{Meaning of FB output Diagnose:}
\begin{itemize}
	\item \texttt{0x8xxx}: OK (xxx = status info)
	\item \texttt{0xC0xx}: Initialization error (typically with log entry, application terminated)
	\item \texttt{0xC1xx}: Own detected error
	\item \texttt{0xC2xx}: Error detected by F-Device
\end{itemize}

\subsubsection{PROFIsafe}

Different behavior regarding FB diagnose output:

\begin{itemize}
	\item V2.4: Diagnose word is overwritten by higher priority diagnoses (e.g., from \texttt{0xC0xx} to \texttt{0xC2xx})
	\item V2.6: First detected error remains until it is acknowledged
\end{itemize}

\textbf{Note:} The F-Host only knows two errors: Timeout and CRC error. The CRC error covers all types of initialization and communication errors!

PROFIsafe diagnostics on standard (Safe Control Core only): F-Host outputs are transferred to the standard controller.

\textbf{Example access:}

\textbf{Declaration:}
\begin{lstlisting}
	uiID: UDINT;
	FHostState : ProfinetCommon.F_HostStatus;
\end{lstlisting}

\textbf{Implementation:}
\begin{lstlisting}
	uiID := IoDrvProfinetBase.GetID(Wago666);
	ProfiNetCommon.GetFHostStatus(ID := uiID, F_Status:= FHostState);
\end{lstlisting}

\section*{Download - Boot Application}

\textbf{Note:} Download and boot application handling differs from standard.

On logout, a running application is unloaded and no longer executed. If a boot application exists, it may or may not start!

In the safety login dialog, it must be decided whether to perform only a download or a download and create a boot application.

\textbf{Download with Boot Application} combines two commands (login and create boot application), which can also be executed separately.

\textbf{Recommendation:} First do a normal download. Only when the application is stable should a boot application be created.

\subsection{Starting Boot Application for SafeControl Core}

For Safe Control Core, the user must confirm the start. This requires the \texttt{CODESYS Safe Control Lib} (see diagnostics).

From Safety Extension 4.3.0.0, the state is shown in the project tree and the application status.

Older versions (< 4.3.0.0) do not show this status.

\section{Diagnostics on Standard}

CODESYS Safe Control Service Package with \texttt{CODESYS Safe Control Lib}:

\begin{itemize}
	\item \texttt{FB SafeDevice}: Device information and diagnostics
	\item \texttt{FB SafeApplication}: Information on loaded application and boot app status
	\item \texttt{FB StartBootApp}: Confirms boot application start (status displayed in tree from 4.3.0.0)
\end{itemize}

To start the boot application:

\begin{itemize}
	\item Read \texttt{SafeApplication.BootAppConfirmation}
	\item Provide a ClientID (in the CFG file of the runtime, default: \texttt{16\#ED387206})
\end{itemize}

\textbf{Example Implementation:}
\begin{lstlisting}
	PROGRAM StartSafetyBA
	VAR
	xStartBA : BOOL := FALSE;
	xRestart : BOOL := FALSE;
	fbSafeApp : SafeControl.SafeApplication;
	fbSafeDev : SafeControl.SafeDevice;
	fbStartBA : SafeControl.StartBootApp;
	fbRestart: SafeControl.RestartBootApp;
	END_VAR
	
	VAR CONSTANT
	c_udnClientId : UDINT := 16#ED387206; // Client ID of standard PLC
	END_VAR
	
	fbSafeApp(xEnable := TRUE);
	fbSafeDev(xEnable := TRUE);
	fbStartBA(xExecute := (fbSafeApp.xBootAppConfirmationRequested AND xStartBA),
	szSafetyDeviceFirmware := fbSafeDev.szSafetyDeviceFirmware,
	udnAppId := fbSafeApp.AppInfo.udnAppId,
	udnClientId := c_udnClientId,
	BootAppConfirmation := fbSafeApp.BootAppConfirmation);
	
	fbRestart(xExecute := xRestart);
\end{lstlisting}

\section{Debugging}

Only online commands are supported: Start, Stop, Write, and Force. Forcing is limited to 20 entries.

\section{Device Editors}
\begin{itemize}
	\item Communication
	\item Safety Online Information
	\item Status
	\item Information
\end{itemize}

\section*{Further Limitations}

Library Manager: Users cannot develop their own libraries. Only libraries with external FBs are currently supported.

\subsection{Additional Links}
\begin{itemize}
	\item Online help for Virtual Safe Control: \url{https://content.helpme-codesys.com/de/CODESYS%20Control/_rtsl_scenario_safe_house.html}
	\item Docker installation: \url{https://docs.docker.com/engine/install/debian/}
\end{itemize}






\end{document}
